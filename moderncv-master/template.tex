%% start of file `template.tex'.
%% Copyright 2006-2015 Xavier Danaux (xdanaux@gmail.com), 2020-2021 moderncv maintainers (github.com/moderncv).
%
% This work may be distributed and/or modified under the
% conditions of the LaTeX Project Public License version 1.3c,
% available at http://www.latex-project.org/lppl/.

\documentclass[10pt,letterpaper,roman]{moderncv}        % possible options include font size ('10pt', '11pt' and '12pt'), paper size ('a4paper', 'letterpaper', 'a5paper', 'legalpaper', 'executivepaper' and 'landscape') and font family ('sans' and 'roman')

% Disable microtype's font expansion to fix compatibility issue
\microtypesetup{expansion=false}

% moderncv themes
\moderncvstyle{banking}                             % style options are 'casual' (default), 'classic', 'banking', 'oldstyle' and 'fancy'
\moderncvcolor{burgundy}                               % color options 'black', 'blue' (default), 'burgundy', 'green', 'grey', 'orange', 'purple' and 'red'

% adjust the page margins
\usepackage[scale=0.8]{geometry}

% Add tabularx for flexible tables
\usepackage{tabularx}

% personal data
\name{Brian}{Benchoff}
\title{Electronic / Embedded Engineer}
\address{3033 Rivera St}{94116 San Francisco}{USA}
\phone[mobile]{+1~(215)~690~1415}
\email{benchoff@gmail.com}
\homepage{https://bbenchoff.github.io/}

% Social icons
\social[linkedin]{bbenchoff}
\social[twitter]{bbenchoff}
\social[github]{bbenchoff}
\social[signal]{12156901415}

% This is the key change: Completely redefine makecvtitle and makecvhead
\makeatletter
% First, we need to empty the makecvhead command to prevent double headers
\renewcommand*{\makecvhead}{}

% Define the burgundy color to match moderncv's burgundy
\definecolor{burgundy}{rgb}{0.5,0,0}
\definecolor{gray}{rgb}{.5,.5,.5}


% Then redefine the makecvtitle to our new two-column format
\renewcommand*{\makecvtitle}{%
  % Define top spacing
  \vspace*{-15pt}
  % Create the two-column header using tabularx
  \begin{tabularx}{\textwidth}{@{}X r@{}}
    % Left column: Name and Title
    \begin{minipage}{0.5\textwidth}
      {\Huge\textcolor{burgundy}{\textbf{\@firstname~\@lastname}}}\\[0.4em]
      {\large\color{gray}\textsc{\@title}}
    \end{minipage} &
    % Right column: Contact Info (aligned right)
    \begin{minipage}{0.48\textwidth}
      \raggedleft
      \ifthenelse{\isundefined{\@addressstreet}}{}{\addresssymbol\@addressstreet\\}%
      \ifthenelse{\isundefined{\@addresscity}}{}{\@addresscity\\}%
      % Adding phone directly
      \mobilesymbol+1~(215)~690~1415\\
      \ifthenelse{\isundefined{\@email}}{}{\emailsymbol\emaillink{\@email}\\}%
      \ifthenelse{\isundefined{\@homepage}}{}{\homepagesymbol\httplink{\@homepage}\\}%
      % Social links - individual instead of using collectionloop
      \ifthenelse{\isundefined{\@linkedinsocialsymbol}}{}{\@linkedinsocialsymbol\httplink{linkedin.com/in/\@linkedinsocial}\\}%
      \ifthenelse{\isundefined{\@twittersocialsymbol}}{}{\@twittersocialsymbol\httplink{twitter.com/\@twittersocial}\\}%
      \ifthenelse{\isundefined{\@githubsocialsymbol}}{}{\@githubsocialsymbol\httplink{github.com/\@githubsocial}\\}%
      \ifthenelse{\isundefined{\@signalsocialsymbol}}{}{\@signalsocialsymbol\@signalsocial\\}%
    \end{minipage}
  \end{tabularx}
  % Add space after the header
  \par\vspace{2.5em}
}
\makeatother

%----------------------------------------------------------------------------------
%            content
%----------------------------------------------------------------------------------
\begin{document}
%\begin{CJK*}{UTF8}{gbsn}                          % to typeset your resume in Chinese using CJK
%-----       resume       ---------------------------------------------------------
\makecvtitle

%\section{Education}
%\cventry{2005-2010}{Electronic Media / Broadcasting}{Shippensburg University of Pennsylvania}{Shippensburg, PA}{\textit{B.A. }}{}  % arguments 3 to 6 can be left empty
%\cventry{2005-2010}{ Psychology}{Shippensburg University of Pennsylvania}{Shippensburg, PA}{\textit{B.A. }}{}

%\section{Master thesis}
%\cvitem{title}{\emph{Title}}
%\cvitem{supervisors}{Supervisors}xtures
%\cvitem{description}{Short thesis abstract}

\section{Professional Summary}
Versatile Prototype Engineer with 10+ years of experience in hardware development and embedded systems. Expertise in designing and fabricating functional prototypes using various manufacturing techniques. Skilled in embedded programming and mechanical CAD. Passionate about transforming innovative concepts into functional products through rapid prototyping and iterative design.

\section{Experience}
%\subsection{Vocational}
\cventry{2022-2025}{Electronic Engineer}{Span.io}{San Francisco}{}{Developed novel data acquisition system for thermocouple measurement, reducing costs by 90. Created comprehensive data acquisition web interface using Flask/React, enabling real-time monitoring for 200+ sensor channels. Designed and fabricated specialized test rigs using KiCad, 3D printing, and traditional manufacturing techniques. Implemented Vehicle-to-Grid (V2G) home backup power solution utilizing Nissan Leaf with CHAdeMO interface, providing 4kW emergency power capability. Executed prototype work using CNC (HAAS, Shapeoko), 3D Printing (Stratasys PolyJet, Filament), laser cutting, and fiber laser technologies.\newline}
\cventry{2016-2022}{Embedded Engineer, Product Designer}{Self-Employed}{San Francisco}{}{Designed, built, and sold consumer electronics. This included 3D CAD Fusion360, AutoCAD, PCB design Eagle, KiCAD, Design for Manufacturability and Design for Assembly. Closely integrated with PCB assembly, up to and including running pick and place machines. Fabrication of 3D printed and injection molded parts in plastic and silicone. Designed, marketed, and sold several successful products.\newline}
\cventry{2018-2022}{Content Specialist}{Supplyframe}{Pasadena}{}{Produced electronic design and engineering content, engaged with engineers regarding new products. Responsible for hardware projects, PCB \& firmware design. 3D modeling, injection molded and 3D printed plastic and silicone.\newline{}}
\cventry{2011-2018}{Editor}{Hackaday}{Pasadena}{}{Wrote, edited, produced content for weblog Hackaday. Designed hardware products and projects.\newline{}}

%\section{Languages}
%\cvitemwithcomment{Language 1}{Skill level}{Comment}
%\cvitemwithcomment{Language 2}{Skill level}{Comment}
%\cvitemwithcomment{Language 3}{Skill level}{Comment}
%\cvitemwithcomment{Language 4}{Skill level}{Comment}

\section{Skills}
\cvdoubleitem{Languages}{C, C++, Python, \LaTeX, SQL}{Graphic}{Adobe Photoshop, Illustrator, Premiere}
\cvdoubleitem{Mechanical CAD}{Fusion360, AutoCAD, \newline OpenSCAD}{Platforms}{x86, 8085, AVR, ARM Cortex-M (M0 \& M4), RP2040/2350 PIO, Linux SoCs (Microchip, Allwinner)}
\cvdoubleitem{Electronic CAD}{Altium, Eagle, KiCAD}{Misc}{Microsoft Office, 3D Printing, \newline Rapid Prototyping, Industrial Design}
\cvdoubleitem{Embedded}{I2C, SPI, Serial, Parallel interfaces, USB, USB-C, HDMI, PCIe, eMMC}{Embedded Linux}{Buildroot, Yocto}




%\section{Skill matrix}
%\cvitem{Skill matrix}{Alternatively, provide a skill matrix to show off your skills}
%% Skill matrix as an alternative to rate one's skills, computer or other. 

%% Adjusts width of skill matrix columns. 
%% Usage \setcvskillcolumns[<width>][<factor>][<exp_width>]
%% <width>, <exp_width> should be lengths smaller than \textwidth, <factor> needs to be between 0 and 1.
%% Examples:
% \setcvskillcolumns[5em][][]%    adjust first column. Same as \setcvskillcolumns[5em]
% \setcvskillcolumns[][0.45][]%   adjust third (skill) column. Same as \setcvskillcolumns[][0.45]
% \setcvskillcolumns[][][\widthof{``Year''}]%     adjust fourth (years) column.
% \setcvskillcolumns[][0.45][\widthof{``Year''}]%
% \setcvskillcolumns[\widthof{``Languag''}][0.48][]
% \setcvskillcolumns[\widthof{``Languag''}]%

%% Adjusts width of legend columns. Usage \setcvskilllegendcolumns[<width>][<factor>]
%% <factor> needs to be between 0 and 1. <width> should be a length smaller than \textwidth
%% Examples:
% \setcvskilllegendcolumns[][0.45]
% \setcvskilllegendcolumns[\widthof{``Legend''}][0.45]
% \setcvskilllegendcolumns[0ex][0.46]% this is usefull for the banking style

%% Add a legend if you are using \cvskill{<1-5>} command or \cvskillentry
%% Usage \cvskilllegend[*][<post_padding>][<first_level>][<second_level>][<third_level>][<fourth_level>][<fifth_level>]{<name>}
% \cvskilllegend % insert default legend without lines
%\cvskilllegend*[1em]{}% adjust post spacing
% \cvskilllegend*{Legend}%  Alternatively add a description string
%% adjust the legend entries for other languages, here German
% \cvskilllegend[0.2em][Grundkenntnisse][Grundkenntnisse und eigene Erfahrung in Projekten][Umfangreiche Erfahrung in Projekten][Vertiefte Expertenkenntnisse][Experte\,/\,Spezialist]{Legende}

%% Alternative legend style with the first three skill levels in one column
%% Usage \cvskillplainlegend[*][<post_padding>][<first_level>][<second_level>][<third_level>][<fourth_level>][<fifth_level>]{<name>}
% \setcvskilllegendcolumns[][0.6]%  works for classic, casual, banking
% \setcvskilllegendcolumns[][0.55]%  works better for oldstyle and fancy
% \cvskillplainlegend{}
% \cvskillplainlegend[0.2em][Grundkenntnisse][Grundkenntnisse und eigene Erfahrung in Projekten][Umfangreiche Erfahrung in Projekten][Vertiefte Expertenkenntnisse][Experte/Guru]{Legende}

%% Add a head of the skill matrix table with descriptions.
%% Usage \cvskillhead[<post_padding>][<Level>][<Skill>][<Years>][<Comment>]%
%\cvskillhead[-0.1em]%   this inserts the standard legend in english and adjust padding
%% Adjust head of the skill matrix for other languages
% \cvskillhead[0.25em][Level][F\"ahigkeit][Jahre][Bemerkung]

%% \cvskillentry[*][<post_padding>]{<skill_cathegory>}{<0-5>}{<skill_name>}{<years_of_experience>}{<comment>}% 
%% Example usages:
%\cvskillentry*{Language:}{3}{Python}{2}{I'm so experienced in Python and have realised a million projects. At least.}
%\cvskillentry{}{2}{Lilypond}{14}{So much sheet music! Man, I'm the best!}
%\cvskillentry{}{3}{\LaTeX}{14}{Clearly I rock at \LaTeX}
%\cvskillentry*{OS:}{3}{Linux}{2}{I only use Archlinux btw}% notice the use of the starred command and the optional 
%\cvskillentry*[1em]{Methods}{4}{SCRUM}{8}{SCRUM master for 5 years}
%% \cvskill{<0-5>} command
% \cvitem{\textbackslash{cvskill}:}{Skills can be visually expressed by the \textbackslash{cvskill} command, e.g. \cvskill{2}}

%\section{Interests}
%\cvitem{hobby 1}{Description}
%\cvitem{hobby 2}{Description}
%\cvitem{hobby 3}{Description}

%\section{Extra 1}
%\cvlistitem{Item 1}
%\cvlistitem{Item 2}
%\cvlistitem{Item 3. This item is particularly long and therefore normally spans over several lines. Did you notice the indentation when the line wraps?}

%\section{Extra 2}
%\cvlistdoubleitem{Item 1}{Item 4}
%\cvlistdoubleitem{Item 2}{Item 5\cite{book2}}
%\cvlistdoubleitem{Item 3}{Item 6. Like item 3 in the single column list before, this item is particularly long to wrap over several lines.}

%\section{References}
%\begin{cvcolumns}
%\cvcolumn{Category 1}{\begin{itemize}\item Person 1\item Person 2\item Person 3\end{itemize}}
 % \cvcolumn{Category 2}{Amongst others:\begin{itemize}\item Person 1, and\item Person 2\end{itemize}(more upon request)}
 % \cvcolumn[0.5]{All the rest \& some more}{\textit{That} person, and \textbf{those} also (all available upon request).}
%end{cvcolumns}

% Publications from a BibTeX file without multibib
%  for numerical labels: \renewcommand{\bibliographyitemlabel}{\@biblabel{\arabic{enumiv}}}% CONSIDER MERGING WITH PREAMBLE PART
%  to redefine the heading string ("Publications"): \renewcommand{\refname}{Articles}
%\nocite{*}
%\bibliographystyle{plain}
%\bibliography{publications}                        % 'publications' is the name of a BibTeX file

% Publications from a BibTeX file using the multibib package
%\section{Publications}
%\nocitebook{book1,book2}
%\bibliographystylebook{plain}
%\bibliographybook{publications}                   % 'publications' is the name of a BibTeX file
%\nocitemisc{misc1,misc2,misc3}
%\bibliographystylemisc{plain}
%\bibliographymisc{publications}                   % 'publications' is the name of a BibTeX file


\end{document}
%-----       letter       ---------------------------------------------------------
% recipient data
\recipient{Company Recruitment team}{Company, Inc.\\123 somestreet\\some city}
\date{January 01, 1984}
\opening{Dear Sir or Madam,}
\closing{Yours faithfully,}
\enclosure[Attached]{curriculum vit\ae{}}          % use an optional argument to use a string other than "Enclosure", or redefine \enclname
\makelettertitle

Lorem ipsum dolor sit amet, consectetur adipiscing elit. Duis ullamcorper neque sit amet lectus facilisis sed luctus nisl iaculis. Vivamus at neque arcu, sed tempor quam. Curabitur pharetra tincidunt tincidunt. Morbi volutpat feugiat mauris, quis tempor neque vehicula volutpat. Duis tristique justo vel massa fermentum accumsan. Mauris ante elit, feugiat vestibulum tempor eget, eleifend ac ipsum. Donec scelerisque lobortis ipsum eu vestibulum. Pellentesque vel massa at felis accumsan rhoncus.

Suspendisse commodo, massa eu congue tincidunt, elit mauris pellentesque orci, cursus tempor odio nisl euismod augue. Aliquam adipiscing nibh ut odio sodales et pulvinar tortor laoreet. Mauris a accumsan ligula. Class aptent taciti sociosqu ad litora torquent per conubia nostra, per inceptos himenaeos. Suspendisse vulputate sem vehicula ipsum varius nec tempus dui dapibus. Phasellus et est urna, ut auctor erat. Sed tincidunt odio id odio aliquam mattis. Donec sapien nulla, feugiat eget adipiscing sit amet, lacinia ut dolor. Phasellus tincidunt, leo a fringilla consectetur, felis diam aliquam urna, vitae aliquet lectus orci nec velit. Vivamus dapibus varius blandit.

Duis sit amet magna ante, at sodales diam. Aenean consectetur porta risus et sagittis. Ut interdum, enim varius pellentesque tincidunt, magna libero sodales tortor, ut fermentum nunc metus a ante. Vivamus odio leo, tincidunt eu luctus ut, sollicitudin sit amet metus. Nunc sed orci lectus. Ut sodales magna sed velit volutpat sit amet pulvinar diam venenatis.

Albert Einstein discovered that $e=mc^2$ in 1905.

\[ e=\lim_{n \to \infty} \left(1+\frac{1}{n}\right)^n \]

\makeletterclosing

%\clearpage\end{CJK*}                              % if you are typesetting your resume in Chinese using CJK; the \clearpage is required for fancyhdr to work correctly with CJK, though it kills the page numbering by making \lastpage undefined



%% end of file `template.tex'.

